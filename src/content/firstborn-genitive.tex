\section{πρωτότοκος-genitive Sequences}

The function of the genitives 

\begin{table}
\setstretch{1}
\begin{tabular}[l]{*3l}
	\toprule
	\textbf{Reference} & \textbf{Parts of speech} & \textbf{Genitive function} \\
	\midrule

	Col 1:15 & adjective noun & genitive of subordination \\
	\multicolumn{3}{p{\linewidth}}{ὅς ἐστιν εἰκὼν τοῦ θεοῦ τοῦ ἀοράτου, \textbf{πρωτότοκος πάσης κτίσεως,}} \\
	\midrule
	
	Heb 12:23 & verb & attributive genitive \\
	\multicolumn{3}{p{\linewidth}}{καὶ ἐκκλησίᾳ \textbf{πρωτοτόκων ἀπογεγραμμένων} ἐν οὐρανοῖς, καὶ κριτῇ θεῷ πάντων, καὶ πνεύμασι δικαίων τετελειωμένων,}\\
	\midrule

	Rev 1:5 & article adjective & genitive of separation \\
	\multicolumn{3}{p{\linewidth}}{καὶ ἀπὸ Ἰησοῦ Χριστοῦ, ὁ μάρτυς ὁ πιστός, ὁ \textbf{πρωτότοκος τῶν νεκρῶν} καὶ ὁ ἄρχων τῶν βασιλέων τῆς γῆς. Τῷ ἀγαπῶντι ἡμᾶς καὶ λύσαντι ἡμᾶς ἐκ τῶν ἁμαρτιῶν ἡμῶν ἐν τῷ αἵματι αὐτοῦ—}\\

	\bottomrule
\end{tabular}
\label{t1}
\caption{Each row indicates the verse reference, the parts of speech representing the genitive that immediately follows πρωτότοκος, the function of the genitive, and the Greek text of the reference below with the specific sequence in boldface.}
\end{table}