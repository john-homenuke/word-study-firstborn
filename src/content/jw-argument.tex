\section{The Jehovah's Witness Argument}

Before we can debate the meaning of firstborn in Col 1:15, we must establish the opposing argument. The NWT translates Colossians 1:15--20 as follows:

\begin{quote}
	He is the image of the invisible God, the firstborn of all creation; \textsuperscript{16} because by means of him all other things were created in the heavens and on the earth, the things visible and the things invisible, whether they are thrones or lordships or governments or authorities. All other things have been created through him and for him. \textsuperscript{17} Also, he is before all other things, and by means of him all other things were made to exist, \textsuperscript{18} and he is the head of the body, the congregation. He is the beginning, the firstborn from the dead, so that he might become the one who is first in all things; \textsuperscript{19} because God was pleased to have all fullness to dwell in him, \textsuperscript{20}  and through him to reconcile to himself all other things by making peace through the blood he shed on the torture stake, whether the things on the earth or the things in the heavens.
\end{quote}

The most complete argument on the interpretation of Col 1:15 published by the \emph{Watchtower Bible and Tract Society of Pennsylvania} (hereafter \emph{Watchtower Society}) is that of the April 8, 1979 edition of \emph{Awake!}.\autocite{awake1979firstborn} In reference to Col 1:15--18, the author states,

\begin{quote}
	Here we find that the Greek words for both ``first-born'' (protótokos) and ``beginning'' (arkhé) describe Jesus as the first one of a group of class, ``the body, the church,'' and therefore he has preeminence in this respect. He also has preeminence in being the first one resurrected to endless life from among all the human dead.---1 Cor. 15:22, 23.

	The same Greek words occur in the Greek Septuagint translation at Genesis 49:3: “Ruben, thou art my first-born [protótokos], thou my strength, and the first [arkhé, “beginning”] of my children.” (Compare Deuteronomy 21:17, Septuagint.) From such Biblical statements it is reasonable to conclude that the Son of God is the firstborn of all creation in the sense of being the first of God’s creatures. In fact, Jesus refers to himself as “the beginning [arkhé] of God’s creation.” (Rev. 3:14, CB) The New World Translation renders the phrase in this verse: “the beginning of the creation by God.”
\end{quote}

In v.\ 15, πάσης κτίσεως (``of all creation'') is interpreted as a partitive genitive. That is, the head noun πρωτότοκος (``firstborn'') is a part of the genitive referent (``all creation''). Hence, Jesus is the first created thing. It follows then that the next 4 instances of πᾶς (``all'') and the instance in v.\ 20 do not strictly mean ``all things,'' but ``all other things.'' Jesus is also the ``beginning'' and the ``firstborn'' among those who rise from the dead. Two examples from the Septuagint (Gen 49:3 and Deut 21:16) are provided that highlight similar usages of πρωτότοκος and ἀρχή in which the referents are the first of a group or class. ἀρχή in Rev 3:14 is also interpreted in this same way.

The author then presents a Trinitarian view on the meaning of ἀρχή in these instances. He quotes Henry Alford (1810--1871) who, the author insinuates, concedes that the simple meaning of ἀρχή indicates that Christ is the first created being.\autocite{alfordgreektestament} As another example, he quotes William Nicoll (1851--1923) who holds to the ``source'' or ``origin'' interpretation of ἀρχή in Rev 3:14, but dismisses his argument on the basis that ``The inspired Bible writers, however, never borrowed ideas from Greek philosophy.''

The author shows that πᾶς can imply ``all other'' with a number of examples to support the original premise that Christ as firstborn is a part of creation. If he is a part of creation, he cannot be Almighty God. This is further supported by the fact that ``The Scriptures repeatedly portray him as in a position subordinate to God.''